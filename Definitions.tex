\documentclass[12pt,a4paper]{article}
\usepackage[utf8]{inputenc}
\usepackage[russian]{babel}
\usepackage[OT1]{fontenc}
\usepackage{amsmath}
\usepackage{amsfonts}
\usepackage{amssymb}
\title{Определения}
\begin{document}
\begin{center}
Билет 1
\end{center}

Упорядоченная пара --- это двухэлементное семейство, где множеством индексов является \{1, 2\}.

Декартовым или прямым произведением множест в X и У называется множество всех упорядоченных пар, таких что первый элемент пары принадлежит X, а второй --- Y.

\begin{center}
Билет 2
\end{center}

TODO

\begin{center}
Билет 3
\end{center}

Определение матматической индукции и бинома Ньютона очевидны.

Индуктивное множество --- $1 \in M$, $\forall x \in M \Rightarrow x+1 \in M$.

$\mathbb{N}$ --- минимальное по включению индуктивное подмножество $\mathbb{R}$. 

\begin{center}
Билет 4
\end{center}

Множество $E \subset \mathbb{R}$ называется \textit{ограниченным сверху/снизу}, если $\exists M \in \mathbb{R}: x \leq M$ ($x \geq M$), $x \in E$. Множество $E$ ограниченное, если оно ограничено и сверху, и снизу.

\end{document}
\documentclass[12pt,a4paper]{article}
\usepackage[utf8]{inputenc}
\usepackage[russian]{babel}
\usepackage[OT1]{fontenc}
\usepackage{amsmath}
\usepackage{amsfonts}
\usepackage{xcolor}
\usepackage{amsthm}
\usepackage{hyperref}
\usepackage{amssymb}
\usepackage{showkeys}
\newtheorem*{theorem*}{Теорема}
\newtheorem*{lemma*}{Лемма}
\newtheorem*{conseq*}{Следствия}
\title{Определения}
\begin{document}	

\begin{center}
Билет 1
\end{center}

Мне лень писать законы де Моргана :(

Упорядоченная пара --- это двухэлементное семейство, где множеством индексов является \{1, 2\}.

Декартовым или прямым произведением множест в X и У называется множество всех упорядоченных пар, таких что первый элемент пары принадлежит X, а второй --- Y.

\begin{center}
Билет 2
\end{center}

Аксиомы вещественных чисел:
\begin{enumerate}
\item ассоциативность сложения $(x+y)+z=x+(y+z)$
\item коммутативность сложения $x+y=y+x$
\item существование нейтрального элемента $\exists 0: x+0=0+x=x$
\item существование обратного элемента $\exists (-x): x+(-x)=(-x)+x=0$
\item ассоциативность умножения $(xy)z=x(yz)$
\item коммутативность умножения $xy=yx$
\item существование нейтрального элемента $\exists 1: x \cdot 1 = 1 \cdot x = x$
\item существование обратного элемента $\exists x^{-1}, x \ne 0: x \cdot x^{-1}= x^{-1} \cdot x = 1$
\item дистрибутивность умножения относительно сложения $x(y+z)=xy+yz$
\end{enumerate}

Так же есть аксиомы Архимеда и Кантора.

\begin{center}
Билет 3
\end{center}

Определение матматической индукции и бинома Ньютона очевидны.

Индуктивное множество --- $1 \in M$, $\forall x \in M \Rightarrow x+1 \in M$.

$\mathbb{N}$ --- минимальное по включению индуктивное подмножество $\mathbb{R}$. 

\begin{center}
Билет 4
\end{center}

Множество $E \subset \mathbb{R}$ называется \textit{ограниченным сверху/снизу}, если $\exists M \in \mathbb{R}: x \leq M$ ($x \geq M$), $x \in E$. Множество $E$ ограниченное, если оно ограничено и сверху, и снизу.

$M$ -- максимум (минимум) множества $E \subset \mathbb{R}$, если $M \in E$ и $M \geq x$ ($M \leq x$), $x \in E$. Максимум $E$ --- $\max E$, минимум $E$ --- $\min E$.

\begin{theorem*}[Существование максимума и минимума конечного множества]
\label{4.1}
Во всяком конечном подмножестве $\mathbb{R}$ есть наибольший и наименьний элемент.

Доказательство по индукции. $\blacksquare$
\end{theorem*}

\begin{conseq*}
\label{4.2}
Во всяком непустом ограниченном сверху (снизу) подмножестве $\mathbb{Z}$ есть наибольший (наименьший) элемент.
\newline
В непустом подмножестве $\mathbb{N}$ есть наименьший элемент.
\end{conseq*}

\begin{center}
Билет 5
\end{center}

Наибольшее целое число не превосходящее $x \in \mathbb{R}$ называется целой частью числа $x$ и обозначается $[x]$.

$[x] \leq x < [x] + 1$, $x - 1 < [x] \leq x$.

\begin{theorem*}[Плотность множества рациональных чисел]
\label{5.1}
Во всяком интервале есть рациональное число.

Для доказательства возьмём $a, b \in \mathbb{R}$, $a < b$. Тогда $\frac{1}{b-a} > 0$ и по аксиоме Архимеда $\exists\  n \in \mathbb{N}:\ n > \frac{1}{b-a}$, то есть $\frac{1}{n} < b-a$. Возьмём $c=\frac{[na]+1}{n}$. Тогда $c \in \mathbb{Q}$ и
\newline 
$$
c \leq \frac{na+1}{n}=a+\frac{1}{n}<a+(b-a)=b,
$$
$$
c > \frac{na-1+1}{n}=a
$$ $\blacksquare$
\end{theorem*}

\begin{conseq*}
\label{5.2}
Во всяком интервале бесконечно много рационаьных чисел.
\end{conseq*}

\begin{center}
Извилистая дорога
\end{center}

Пусть $f: X \rightarrow Y$, $A \subset X$. Образ множества $A$ при отображении $f$ --- это $$f(A)=\{y \in Y:\ \exists x \in A\ f(x)=y\}.$$

Пусть $f: X \rightarrow Y$, $B \subset Y$. Прообраз множества $B$ при отображении $f$ --- это $$f^{-1}(B)=\{x \in X: f(x) \in B\} .$$

Пусть $f: X \rightarrow Y$. Если $f(X) = Y$, то $f$ --- сюрьекция. Другими словами, для любого $Y$ есть решение в $X$.

Пусть $f: X \rightarrow Y$. Если для любых различных элементов из $X$ их образы различны, то $f$ --- инъекция.

Биекция = инъекция + сюрьекция.

Пусть $f: X \rightarrow Y$ и $f$ обратимо. Тогда обратное отображение к $f$ --- это отображение $f^{-1}$, которое каждому элементу $y$ из $f(X)$ сопоставляет (единственное) значение $x$ из $X$, для которого $f(x)=y$.

Пусть $f: X \rightarrow Y$ и $g: Y \rightarrow Z$, $f(X) \subset Y$. Тогда композиция $(g \circ f)(x)$ есть $h(x) = g(f(x))$, $x \in X$. $g$ --- внешнее отображение, $f$ --- внутреннее отображение. $f \circ g \ne g \circ f$.

\begin{center}
Билет 6
\end{center}

Два множества эквивалентны, если между ними можно провести биекцию.

Множество называется счётным, если оно эквивалентно множеству натуральных чисел.

\begin{theorem*}
\label{6.1}
Всякое бесконечное множество содержит счётное подмножество

Пусть $A$ --- бесконечное множество. В нём есть $a_1$, $A \setminus \{a_1\}$ бесконечно, поэтому в нём есть $a_2$. $A \setminus \{a_1, a_2\}$ бесконечно, поэтому в нём есть $a_3$  т.д. Новое множество $B = \{a_1, a_2, a_3, ...\}$ будет счётным по построению. $\blacksquare$
\end{theorem*}

\begin{theorem*}
\label{6.2}
Всякое бесконечное подмножество счётного подмножества счётно: $A$ счётно, $B \subset A$ и $B$ бесконечно, то $B$ --- счётно.

Расположим элементы $A$ в виде последовательности
$$
A = \{ a_1, a_2, a_3, \ldots \}
$$
Будем нумеровать элементы $B$ в порядке их появления в этой последовательности . Тем самым каждый элемент $B$ будет занумерован ровно один раз и, так как множество $B$ бесконечно, для нумерации будет использован весь натуральный ряд . $\blacksquare$
\end{theorem*}

Не более чем счётное множество --- пустое, конечное или счётное множество.

\begin{center}
Билет 7
\end{center}

\begin{lemma*}
\label{7.1}
Пусть элементы множества $A$ расположены в виде бесконечной в обоих направлениях матрице
$$
A=
\begin{pmatrix}
a_{11} & a_{12} & a_{13} & \ldots \\
a_{21} & a_{22} & a_{23} & \ldots \\
a_{31} & a_{32} & a_{33} & \ldots \\ 
\ldots & \ldots & \ldots & \ldots
\end{pmatrix}
$$
тогда $A$ счётно (для доказательства перечисляй элементы по дополнительным диагоналям).
\end{lemma*}

По лемме $\mathbb{N} \times \mathbb{N}$ --- счётно.

\begin{theorem*}
\label{7.2}
Не более чем счётное объединение не более чем счётных множеств не более чем счётно.

Пусть $B=\bigcup\limits_{k=1}^{n} A_k$ или $B=\bigcup\limits_{k=1}^{\infty} A_k$, $A_k$ не более чем счётно. Запишем элементы $A_k \setminus \bigcup\limits_{i=1}^{k-1} A_i$ в $k$-ю строку матрицы. Таким образом все клетки окажутся записаны в матрицу и по лемме выше мы получили, что $B$ эквивалентно некоторому подмножеству $\mathbb{N} \times \mathbb{N}$. $\blacksquare$
\end{theorem*}

\begin{center}
Билет 8
\end{center}

\begin{theorem*}
\label{8.1}
Множество рациональных чисел счётно.

Обозначим 
$$
\mathbb{Q}_{+} = \{x \in \mathbb{Q}: x>0\}, \mathbb{Q}_{-} = \{x \in \mathbb{Q}: x < 0\}
$$ 
При всех $q \in \mathbb{N}$ множество $\mathbb{Q}_{q} = \left\lbrace \frac{1}{q}, \frac{2}{q}, \frac{3}{q}, \ldots \right\rbrace$ счётно. По теореме 7.2 и $\mathbb{Q}_{+} = \bigcup_{q=1}^{\infty} Q_p$ счётно. Очевидно, что $\mathbb{Q}_{-} \sim \mathbb{Q}_{+}$. Снова по теореме 7.2 множество $\mathbb{Q} = \mathbb{Q}_{-} \cup \mathbb{Q}_{+} \cup {0}$ счётно. $\blacksquare$
\end{theorem*}

\begin{center}
Билет 9
\end{center}

\begin{theorem*}[Несчётность отрезка]
\label{9.1}
Отрезок $[0, 1]$ несчётен.

Допустим противное: пусть отрезок $[0, 1]$ счётен, тогда все числа в нём можно расположить в виде последовательности
$$
[0, 1] = \{ x_1, x_2, x_3, \ldots \}
$$
Разобъём отрезок $[0, 1]$ на $3$ равных отрезка $[0, \frac{1}{3}]$, $[\frac{1}{3}, \frac{2}{3}]$ и $[\frac{2}{3}, 1]$. Обозначим через $[a_1, b_1]$ тот из них, который не содержит точки $x_1$. Далее разобъём $[a_1, b_1]$ на 3 отрезка по алгоритму выше и обозначим через $[a_2, b_2]$ тот, который не содержит $x_2$. Продолжим этот процесс неограниченно. В результате получим последовательность вложенных отрезков $\{ [a_n, b_n] \}_{n=1}^{\infty}$. По аксиоме о вложенных отрезка существует точка $x^{*}$, принадлежащая всем отрезкам $[a_n, b_n]$. Также, $x^{*} \in [0, 1]$. Но тогда существует $m: x^{*}=x_m$ (по условию). По построению $x_m \not\in [a_m, b_m] \Rightarrow x^{*} \not\in [a_m, b_m]$, что противоречит принадлежности $x^{*}$ всем отрезкам. $\blacksquare$
\end{theorem*}

\end{document}